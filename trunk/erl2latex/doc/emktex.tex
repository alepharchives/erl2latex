\documentclass[a4paper,12pt]{article}
\newenvironment{mylisting}
{\begin{list}{}{\setlength{\leftmargin}{1em}}\item\scriptsize\bfseries}
{\end{list}}

\begin{document}
\title{emktex: erl2latex Build Script}
\maketitle
Copyright (c) 2008 Ulf Wiger, John Hughes\footnote{
\tiny{The MIT License

Copyright (c) 2008 Ulf Wiger <ulf@wiger.net>,
John Hughes <john.hughes@quviq.com>

Permission is hereby granted, free of charge, to any person obtaining a
copy of this software and associated documentation files (the "Software"),
to deal in the Software without restriction, including without limitation
the rights to use, copy, modify, merge, publish, distribute, sublicense,
and/or sell copies of the Software, and to permit persons to whom the
Software is furnished to do so, subject to the following conditions:

The above copyright notice and this permission notice shall be included in
all copies or substantial portions of the Software.

THE SOFTWARE IS PROVIDED "AS IS", WITHOUT WARRANTY OF ANY KIND, EXPRESS OR
IMPLIED, INCLUDING BUT NOT LIMITED TO THE WARRANTIES OF MERCHANTABILITY,
FITNESS FOR A PARTICULAR PURPOSE AND NONINFRINGEMENT. IN NO EVENT SHALL
THE AUTHORS OR COPYRIGHT HOLDERS BE LIABLE FOR ANY CLAIM, DAMAGES OR OTHER
LIABILITY, WHETHER IN AN ACTION OF CONTRACT, TORT OR OTHERWISE, ARISING 
FROM, OUT OF OR IN CONNECTION WITH THE SOFTWARE OR THE USE OR OTHER
DEALINGS IN THE SOFTWARE.
}}
\begin{mylisting}
\begin{verbatim}
#!/usr/bin/env escript
%%! -smp enable
\end{verbatim}
\end{mylisting}

\begin{mylisting}
\begin{verbatim}
-include_lib("kernel/include/file.hrl").

main(Args) ->
    try code:add_path(path()),
        chk_args(Args) of
        {F, Out} ->
            case erl2latex:file(F, [{outdir,Out}]) of
                ok ->
                    return_ok();
                Error ->
                    rpt("Error (~p)~n", [Error]),
                    return_error()
            end;
        OtherArgs ->
            rpt("Bad args (~p).~n"
                "Usage: emktex Src Target~n", [OtherArgs]),
            return_error()
    catch
        error:E ->
            rpt("Exception (~p)~n", [E]),
            return_error()
    end.

chk_args(Args) ->
    chk_args(Args, {undefined,undefined}).

chk_args(["-o",Out|Args], {Src,_}) ->
    chk_args(Args, {Src,Out});
chk_args([Src|Args], {_,Out}) ->
    chk_args(Args, {Src,Out});
chk_args([], {Src,Out}) ->
    {Src,Out}.

rpt(Msg,Args) ->
    io:fwrite("emktex: " ++ Msg, Args).

return_error() ->
    init:stop(-1).

return_ok() ->
    init:stop(0).
\end{verbatim}
\end{mylisting}

This fairly elaborate function tries to find the ebin/ directory
where erl2latex.beam resides. It is designed to work even if the emktex
script is started from a symlink to the original.
\begin{mylisting}
\begin{verbatim}
path() ->
    Script = escript:script_name(),
    Src = src_name(Script),
    Abs = filename:absname(Src),
    filename:join(dir(dir(Abs)), "ebin").

src_name(F) ->
    case file:read_link_info(F) of
        {ok, #file_info{type = symlink}} ->
            {ok, Src} = file:read_link(F),
            src_name(Src);
        {ok, #file_info{type = regular}} ->
            F
    end.

dir(F) ->
    filename:dirname(F).
\end{verbatim}
\end{mylisting}


\end{document}
