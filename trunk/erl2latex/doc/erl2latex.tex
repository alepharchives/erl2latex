\documentclass[a4paper,12pt]{article}
\newenvironment{mylisting}
{\begin{list}{}{\setlength{\leftmargin}{1em}}\item\scriptsize\bfseries}
{\end{list}}

\begin{document}
\title{erl2latex: Literal Erlang Programming}
\author{Ulf Wiger <ulf@wiger.net>}
\maketitle

Copyright (c) 2008 Ulf Wiger, John Hughes\footnote{
\tiny{The MIT License

Copyright (c) 2008 Ulf Wiger <ulf@wiger.net>,
John Hughes <john.hughes@quviq.com>

Permission is hereby granted, free of charge, to any person obtaining a
copy of this software and associated documentation files (the "Software"),
to deal in the Software without restriction, including without limitation
the rights to use, copy, modify, merge, publish, distribute, sublicense,
and/or sell copies of the Software, and to permit persons to whom the
Software is furnished to do so, subject to the following conditions:

The above copyright notice and this permission notice shall be included in
all copies or substantial portions of the Software.

THE SOFTWARE IS PROVIDED "AS IS", WITHOUT WARRANTY OF ANY KIND, EXPRESS OR
IMPLIED, INCLUDING BUT NOT LIMITED TO THE WARRANTIES OF MERCHANTABILITY,
FITNESS FOR A PARTICULAR PURPOSE AND NONINFRINGEMENT. IN NO EVENT SHALL
THE AUTHORS OR COPYRIGHT HOLDERS BE LIABLE FOR ANY CLAIM, DAMAGES OR OTHER
LIABILITY, WHETHER IN AN ACTION OF CONTRACT, TORT OR OTHERWISE, ARISING 
FROM, OUT OF OR IN CONNECTION WITH THE SOFTWARE OR THE USE OR OTHER
DEALINGS IN THE SOFTWARE.
}}
\section{Introduction}
This module converts an Erlang source file to latex. The latex file
can then be converted to e.g. PDF, using pdflatex or similar tool.

The idea of `literal Erlang programming' is that the source and comments
should read as a good paper. Unlike XML markup, Latex markup is also 
fairly unobtrusive when reading the source directly.
\begin{mylisting}
\begin{verbatim}
-module(erl2latex).

-export([file/1, file/2]).
\end{verbatim}
\end{mylisting}

\section{file/[1,2]}

The interface is:\\
file(Filename [, Options]) -> ok | \{error, Reason\}

Options can be specified either when calling file/[1,2], or 
by adding an attribute, -erl2latex(Options), in the source code.
The attribute will not be included in the latex output.
Options given in the function call will shadow options embedded in
the source code.

mode: normal or included. If \'included\', preamble and document 
begin and end markers are removed if found.
\begin{mylisting}
\begin{verbatim}
-spec file/1 :: (Filename::string()) -> ok.

file(F) ->
    file(F, []).

-type option() :: {documentclass, none | auto | string()}
                  | {mode, normal | included}
                  | {source_listing, auto | string()}.

-spec file/2 :: (Filename::string(), [option()]) -> ok.

file(F, Options) ->
    case file:read_file(F) of
        {ok, Bin} ->
            output(convert_to_latex(Bin, Options), latex_target(F,Options));
        Err ->
            Err
    end.
\end{verbatim}
\end{mylisting}

\section{Conversion to Latex}
Below is the actual conversion function. We separate comments from code,
and convert each block to latex separately. We then insert a preamble,
if not already present, or insert a small formatting macro for the 
source code (if not already defined).
\begin{mylisting}
\begin{verbatim}
convert_to_latex(Bin, Options0) ->
    Parts0 = split_input(binary_to_list(Bin)),
    {Parts, Embedded_options} = embedded_options(Parts0),
    Options = Options0 ++ Embedded_options,
    Mode = proplists:get_value(mode, Options, normal),
    case lists:flatten([convert_part(P) || P <- Parts]) of
        "\\document" ++ _ = Latex0 ->
            {Preamble,Doc} = get_preamble(Latex0),
            [[[Preamble,
               source_listing_setup(Preamble),
               "\\begin{document}\n"] || Mode == normal],
             Doc,
             end_doc(Mode)];
        Latex0 ->
            [[[default_preamble(Options),
               "\\begin{document}\n"] || Mode == normal],
             Latex0,
             end_doc(Mode)]
    end.
\end{verbatim}
\end{mylisting}

If a preamble is present, the $\backslash$begin{document} entry
must also be present. This is how we know where the preamble ends.
\begin{mylisting}
\begin{verbatim}
get_preamble(Str) ->
    get_preamble(Str, []).

get_preamble("\\begin{document}" ++ Rest, Acc) ->
    {lists:reverse("\n" ++ Acc), Rest};
get_preamble([H|T], Acc) ->
    get_preamble(T, [H|Acc]).
\end{verbatim}
\end{mylisting}

The following functions output the default latex preamble and 
document end marker.
\begin{mylisting}
\begin{verbatim}
default_preamble(Options) ->
    [Doc_class,Src_listing] =
        [proplists:get_value(P,Options) || 
            P <- [documentclass, source_listing]],
    document_class(Doc_class) ++ source_listing_setup(Src_listing,"").

document_class(Opt) ->
    if Opt==auto; Opt==undefined ->
            "\\documentclass[a4paper,12pt]{article}\n";
       Opt==none -> "";
       is_list(Opt) -> Opt
    end.

source_listing_setup(Opt,Preamble) ->
    case regexp:first_match(Preamble, "begin{mylisting}") of
        {match,_,_} ->
            [];
        nomatch ->
            source_listing_setup(Opt)
    end.

source_listing_setup(undefined) ->
    source_listing_setup
      ("\\setlength{\\leftmargin}{1em}}"
       "\\item\\scriptsize\\bfseries");
source_listing_setup(Str) when is_list(Str) ->
    ("\\newenvironment{mylisting}\n"
     "{\\begin{list}{}{")
        ++ Str
        ++ ("}\n"
            "{\\end{list}}\n"
            "\n").
    

end_doc() ->
    "\n\\end{document}\n".

end_doc(included) ->
    "";
end_doc(normal) ->
    end_doc().
\end{verbatim}
\end{mylisting}

In this function, we wrap the different `source' and `comment' blocks
appropriately. The weird-looking split between string parts is to keep
pdflatex from tripping on what looks like the end of the verbatim block.
\begin{mylisting}
\begin{verbatim}
convert_part({code,[]}) -> [];
convert_part({code,Lines}) ->
    ["\\begin{mylisting}\n"
     "\\begin{verbatim}\n",
     [expand(L) || L <- normalize(Lines)],
     "\\" "end{verbatim}\n"
     "\\end{mylisting}\n\n"];
convert_part({comment,Lines}) ->
    Lines.

normalize(["\n","\n"|T]) ->  normalize(["\n"|T]);
normalize([H|T])         ->  [H|normalize(T)];
normalize([])            ->  [].
\end{verbatim}
\end{mylisting}

The expand(Line) function expands tabs for better formatting.
\begin{mylisting}
\begin{verbatim}
expand(Line) ->
    expand_tabs(Line).

expand_tabs(Xs) ->
  expand_tabs(0,Xs).

expand_tabs(_N,[]) ->
    [];
expand_tabs(N,[$\t|Xs]) ->
    N1 = 8*(N div 8) + 8,
    [$\s || _ <- lists:seq(N,N1)] ++ expand_tabs(N1,Xs);
expand_tabs(N,[X|Xs]) ->
    [X|expand_tabs(N+1,Xs)].
\end{verbatim}
\end{mylisting}

Following edoc convention, comments are excluded if the first non-space
character following the leading string of \% is another \%,
for example:\\
\%\% \% This comment will be excluded.
\begin{mylisting}
\begin{verbatim}
strip_comment(C) ->
    C1 = strip_percents(C),
    case string:strip(C1, left) of
        "%" ++ _ -> "";
        Stripped ->
            Stripped
    end.

strip_percents("%" ++ C) -> strip_percents(C);
strip_percents(C)        -> C.
\end{verbatim}
\end{mylisting}

\section{Utility Functions}
\begin{mylisting}
\begin{verbatim}
split_input(Txt) ->
    [{T1,Ls} ||
        {T1,Ls} <- 
            [{T,strip_empty(L1)} || {T,L1} <-
                                        group([wrap(L) || L <- lines(Txt)])],
        Ls =/= []].

lines(Str) ->
    lines(Str, []).

lines("\n" ++ Str, Acc) ->
    [lists:reverse([$\n|Acc]) | lines(Str,[])];
lines([H|T], Acc) ->
    lines(T, [H|Acc]);
lines([], Acc) ->
    [lists:reverse(Acc)].
    

wrap("%" ++ S) ->
    {comment, strip_comment(S)};
wrap(S) ->
    {code, S}.

strip_empty(Ls) ->
    Strip = fun(Ls1) -> 
                    lists:dropwhile(fun(L) -> strip_space(L) == [] end, Ls1)
            end,
    lists:reverse(Strip(lists:reverse(Strip(Ls)))).
\end{verbatim}
\end{mylisting}

Remove leading empty lines, even if they contain whitespace.
\begin{mylisting}
\begin{verbatim}
strip_space(L) ->
    lists:dropwhile(fun(C) when C==$\s; C==$\t; C==$\n -> true;
                       (_) -> false
                    end, L).

group([{T,C}|Tail]) ->
    {More,Rest} = lists:splitwith(fun({T1,_C1}) -> T1 == T end, Tail),
    [{T,[C|[C1 || {_,C1} <- More]]} | group(Rest)];
group([]) ->
    [].

latex_target(F, Options) ->
    Target_base = filename:basename(F,".erl") ++ ".tex",
    Outdir = proplists:get_value(outdir, Options, filename:dirname(F)),
    filename:join(Outdir, Target_base).

output(Data, F) ->
    file:write_file(F, list_to_binary(Data)).

embedded_options(Parts) ->
    lists:mapfoldl(
      fun({code,Ls}=Part,Acc) ->
              case scan_for_opts(Ls) of
                  none       -> {Part,Acc};
                  {Opts,Ls1} -> {{code,Ls1}, Acc ++ Opts}
              end;
         (Other, Acc) ->
              {Other, Acc}
      end, [], Parts).

scan_for_opts(Ls) ->
    scan_forms(Ls, [], []).

scan_forms(Ls, Opts0, Acc) ->
    case scan_tokens(Ls) of
        {{ok,[{'-',_},{atom,_,erl2latex},{'(',L}|Ts], _}, Used, Rest} ->
            case erl_parse:parse_term([{'(',L}|Ts]) of
                {ok, Opts} -> scan_forms(Rest, Opts0++Opts, Acc);
                {error,_}  -> scan_forms(Rest, Opts0, Acc ++ Used)
            end;
        {{eof,_}, Used, []} ->
            case Opts0 of 
                [_|_] -> {Opts0, Acc ++ Used};
                []    -> none
            end;
        {_, Used, Rest} ->
            scan_forms(Rest, Opts0, Acc ++ Used)
    end.

scan_tokens([L|Ls]) ->
    scan_tokens(erl_scan:tokens([],L,1), Ls, [L]).

scan_tokens({done,Result,Leftover_chars},Rest,Used) ->
    {Result, lists:reverse(Used), [Leftover_chars|Rest]};
scan_tokens({more, Cont}, Ls, Used) ->
    case Ls of
        [] ->
            {{eof,1}, lists:reverse(Used), []};
        [L|Rest] ->
            scan_tokens(erl_scan:tokens(Cont, L, 1), Rest, [L|Used])
    end.
\end{verbatim}
\end{mylisting}


\end{document}
